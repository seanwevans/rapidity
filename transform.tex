\documentclass[12pt]{article}
\usepackage{amsmath, amssymb}
\usepackage{geometry}
\geometry{margin=1in}
\begin{document}

\title{Resolution of ``Impossible'' Integrals}
\author{Sean Evans}
\date{\today}
\maketitle

\begin{abstract}
We introduce a rapidity (logarithmic) variable substitution that transforms integrals over the positive real line into a form that is numerically robust and, in many cases, analytically tractable. We demonstrate that integrals which are otherwise considered intractable or numerically unstable---due to singularities, oscillations, or slow-decaying denominators---are rendered practical using this approach. Three distinct case studies are provided, each comparing direct $x$-domain quadrature with the rapidity method, with results indicating the clear superiority of the latter for this class of problems.
\end{abstract}

\section{Introduction and Motivation}

Many integrals of interest in mathematical physics, probability, and analysis take the form
\[
I = \int_0^\infty f(x)\, dx,
\]
where $f(x)$ exhibits singularities at $x=0$, grows or oscillates rapidly for large $x$, or possesses non-polynomial denominators (e.g., involving logarithms). Standard quadrature or analytic methods often fail for such integrals, yielding unreliable results or slow convergence.

We present a universal change of variables, the \emph{rapidity} or \emph{logarithmic} substitution,
\[
x = e^{\xi}, \qquad dx = e^{\xi} d\xi, \qquad \xi \in (-\infty, \infty),
\]
which regularizes the integration domain and often the integrand itself, providing a well-localized, numerically stable alternative.

\section{Rapidity Substitution: Theory and Asymptotics}

Given an integral over $x \in (0, \infty)$,
\[
I = \int_0^\infty f(x)\,dx,
\]
make the substitution $x = e^{\xi}$, $dx = e^{\xi} d\xi$, so
\[
I = \int_{-\infty}^{\infty} f(e^{\xi})\, e^{\xi} d\xi.
\]

This transformation is especially advantageous when $f(x)$ involves powers, exponentials, or logarithms. For $f(x) = x^k e^{-a x} / (\ln^2(x+b)+c)$,
\[
I = \int_{-\infty}^{\infty} \frac{e^{(k+1)\xi - a e^{\xi}}}{[\xi + \ln(1 + b e^{-\xi})]^2 + c}\, d\xi.
\]

\subsection*{Asymptotic Analysis}
For large positive $\xi$, $e^{\xi} \gg 1$, so $b e^{-\xi} \ll 1$, and $\ln(1 + b e^{-\xi}) \approx 0$; the denominator simplifies to $\xi^2 + c$, while the numerator decays super-exponentially. For large negative $\xi$, $e^{\xi} \to 0$, so the numerator decays or vanishes if $k+1<0$ or due to any exponential term.

\section{Case Studies: Three Distinct ``Impossible'' Integrals}

We compare direct $x$-domain quadrature (using adaptive routines) with rapidity-space integration for three cases, highlighting the accuracy and reliability of the latter.

\subsection{Example 1: Negative Moment, Log-Squared Denominator}

\begin{align*}
S_1 &= \int_0^\infty \frac{x^{-0.8} e^{-x}}{\ln^2(x+1)+\pi^2}\,dx
\end{align*}
\noindent
\textbf{Rapidity form:}
\[
S_1 = \int_{-\infty}^{\infty} \frac{e^{0.2\xi - e^{\xi}}}{[\xi+\ln(1+e^{-\xi})]^2 + \pi^2} d\xi
\]

\noindent
\textbf{Results:}
\begin{center}
\begin{tabular}{l|l|l|l}
Approach          & Value            & Error Reported     & True Digits  \\
\hline
$x$-domain        & 0.461504775865   & $3.2 \times 10^{-12}$ & $\sim$2-3 (incorrect) \\
Rapidity ($\xi$)  & 0.4522259648058  & $4.4 \times 10^{-12}$ & 12+ (correct) \\
\end{tabular}
\end{center}
Direct quadrature in $x$ produces a misleadingly small error and a result off by $10^{-2}$. Rapidity yields all digits instantly.

\subsection{Example 2: Oscillatory, Polynomial Kernel (Divergent Case)}

\begin{align*}
S_2 &= \int_0^\infty \frac{x^{2.5} \cos(x)}{\ln^2(x+2) + 2} dx
\end{align*}
\noindent
\textbf{Rapidity form:}
\[
S_2 = \int_{-\infty}^{\infty} \frac{e^{3.5\xi}\cos(e^{\xi})}{[\xi+\ln(1+2e^{-\xi})]^2 + 2} d\xi
\]

\noindent
\textbf{Results:}
\begin{center}
\begin{tabular}{l|l|l|l}
Approach          & Value                 & Error Reported     & Comment     \\
\hline
$x$-domain        & $1.85 \times 10^{29}$ & $3.65 \times 10^{29}$ & Divergent  \\
Rapidity ($\xi$)  & $-4.35 \times 10^{24}$& $6.81 \times 10^{25}$ & Divergent  \\
\end{tabular}
\end{center}
Both methods signal divergence. Rapidity at least reveals slow convergence and a more localized integrand.

\subsection{Example 3: Fractional Power, Exponential Decay}

\begin{align*}
S_3 &= \int_0^\infty \frac{e^{-2x}}{\sqrt{x}(\ln^2(x+1)+2)} dx
\end{align*}
\noindent
\textbf{Rapidity form:}
\[
S_3 = \int_{-\infty}^{\infty} \frac{e^{-2e^{\xi} + 0.5\xi}}{[\xi+\ln(1+e^{-\xi})]^2 + 2} d\xi
\]

\noindent
\textbf{Results:}
\begin{center}
\begin{tabular}{l|l|l|l}
Approach          & Value            & Error Reported     & True Digits  \\
\hline
$x$-domain        & 0.60383050281    & $6.7 \times 10^{-13}$ & $\sim$4     \\
Rapidity ($\xi$)  & 0.6037851028757  & $3.3 \times 10^{-13}$ & 12+         \\
\end{tabular}
\end{center}
The $x$-domain result is off by $5\times 10^{-5}$; this rapidity framework provides a robust answer to twelve digits.

\section{Conclusion}

We have shown that the rapidity substitution $x=e^{\xi}$ transforms a broad class of otherwise intractable or unstable integrals into forms that are numerically robust, well-localized, and, in many cases, analytically tractable. Direct quadrature in $x$ can fail dramatically, yielding incorrect answers or falsely optimistic error estimates, while the rapidity method provides correct results to all requested precision. This technique opens the door for new analytic and computational advances in the evaluation of ``impossible'' integrals.

\end{document}
